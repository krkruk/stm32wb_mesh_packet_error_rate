Niniejsza praca zrealizowała postawione założenia zdefiniowane we wstępie -- rozdział~\ref{ch:intro}.
Wprowadziwszy definicję sygnału płynnie wprowadzono zagadnienia związane z~rozpatrywanymi
standardami komunikacji bezprzewodowej.

Zestawiając wprowadzone protokoły na szczególną uwagę zasługują dwa z nich: badany Bluetooth~5
(z~wyróżnieniem Mesh) oraz Thread. Opierając się na wiedzy pochodzącej z~powszechnie
dostępnej literatury, Thread osiągnął znacząco \textit{lepsze} parametry transmisji danych.
Korzyść ta rozumiana jest jako najwyższa przepustowość dla sfragmentowanych danych (Rysunek~\ref{rys:throughput_vs_hops_an1142})
oraz najmniejsze opóźnienia niezależnie od wielkości wiadomości (Rysunek~\ref{rys:latency_vs_payload_an1142}) spośród
testowanych rozwiązań.

Thread nie definiuje warstwy aplikacji, jak to umożliwiają bądź wymuszają pozostałe rozważane standardy.
Czyni to protokół uniwersalnym, niemalże ogólnego przeznaczenia niczym protokoły TCP/IP. Porównanie jest
również nie bez znaczenia, gdyż Thread opiera się właśnie na IPv6, co znów zmniejsza nachylenie
krzywej uczenia się tego standardu i umożliwia niemalże bezpośrednio wpięcie do Internetu i~rozwiązań opartych
o~publiczną chmurę obliczeniową. Protokół ten, będąc oparty o IEEE 802.15.4, nie jest energooszczędny
do tego stopnia, co protokół BLE (Rysunek~\ref{rys:energy_per_packet_dbm_10.4108}). 
Nie czyni go to jednak gorszym a raczej przypisuje go do innych
zastosowań. Potencjalne problemy może stwarzać niewielka popularność tego protokołu wśród oferowanych
na rynku produktów. Jest to niemniej powiązane ze względną nową architekturą danej specyfikacji.
Utrudnieniem, które bezpośrednio wpływa na kompatybilność produktów różnych producentów, jest brak
wprost zdefiniowanej warstwy aplikacji. To wyzwanie adresują projekty, takie jak 
Matter\footnote{\url{https://csa-iot.org/all-solutions/matter/}}
udostępniając specyfikację integrującą rozwiązania oparte o~Thread.

Bluetooth Mesh, pomimo rezultatów wywodzących się z~powszechnej literatury również spełnia swoje
oczekiwania, do których został zaprojektowany. Protokół, kładzie szczególny nacisk na organizację
i~hierarchię transmitowanych wiadomości w~warstwie. Można wystosować wręcz opinię, że został stworzony
do wysyłania \textit{tylko jednego} komunikatu, by wykonać konkretną akcję. Takie twierdzenie można
poprzeć zarówno sposobem routingu i wspomnianej hierarchii wiadomości (modele Mesh), ale również
z biorąc uwagę wyniki reprezentowane na Rysunku~\ref{rys:latency_vs_payload_an1142}. Pojedynczy wysłany
komunikat cechuje się nie gorszym opóźnieniem niż pozostałe badane protokoły. Uwzględniając energooszczędność
tego standardu, czyni go nadzwyczaj interesującą opcją dla konkretnych zastosowań, np. inteligentnego
oświetlenia.

Dalsza część pracy to ćwiczenie praktyczne polegające pomiarze zużycia energii. Zaprezentowano
charakterystyki poboru prądu w czasie dla różnych trybów funkcjonowania urządzenia dla dwóch różnych firmware'ów.
Pierwszy rozważany przypadek oparty o BLE \gls{HRT} jest zgodny z~przeprowadzoną symulacją -- 
Rysunki\ref{rys:cube_pcc_advertising_1ms} oraz~\ref{rys:cube_pcc_connected_master}. Wykonana zgrubna symulacja,
bazująca na domyślnych parametrach wskazywanych przez dokumentację,
wskazywała na średni pobór prądu elektrycznego na poziomie 2,95mA (moc: 9,8mW)~\cite{noauthor_um1718_2022}. Jest to co prawda niemal
5-krotna różnica względem uzyskanych pomiarów -- maksymalny pobór prądu 0,69mA (moc: 2,27mW) -- jest ona jednak na korzyść
eksperymentu. Sugeruje to błąd dokonanych nastawów symulatora.

Przypadek zużycia Mesh wskazuje na zużycie energii węzła klasycznego serwera \textit{Generic OnOff}. Węzeł tego
najczęściej włączony jest do stałego źródła zasilania. Stąd, porównanie z Rysunku~\ref{rys:power_ble_consumption_comparison},
sugeruje na konieczność porównywania węzła typu \textit{LPN}, który w zamyśle -- potwierdzonym z~oficjalną specyfikacją
-- funkcjonuje z~wykorzystaniem zasilania bateryjnego. Słusznym rozwinięciem tego badania jest wykorzystania
\textit{LPN} z~jednosekundowym odświeżaniem swojego stanu (zakładając model \textit{Sensors}). Energooszczędność
BLE i Bluetooth Mesh wprost wynika z optymalizacji wykorzystania radia, co udowadnia choćby porównanie
wyników zaprezentowanych na charakterystykach~\ref{rys:power_ble_hr_fastadv_only_amps} oraz~\ref{rys:power_ble_hr_low_power_adv_only_amps},
gdzie podczas rozgłaszania niskomocowego (zgodnie z~opisywanymi w~danym rozdziale założeniami), radio przez większość
czasu nie pracuje.

Eksperyment PER wskazuje na kilka interesujących zależności. Pierwszą z nich jest wydajność stosu \gls{BT} mikrokontrolera \textit{STM32WB55RG}.
Przed przystąpieniem do formalnego doświadczenia, przeprowadzono szereg prób z częstościami wykonywanych zapytań
znacząco mniejszymi od 100ms --~w~analogii do ciągu Fibonacciego testowano interwały 10, 20, 30, 50 i~80ms. Każdy z~nich wykazywał
tę samą cechę utraty pakietów, jak na przedstawionym Rysunku~\ref{rys:per_to_distance_under_100ms}. Przeważnie,
po kilku-kilkunastu sekundach pracy węzeł bliższy nie odpowiadał na jakiekolwiek komunikaty AT. Po zrestartowaniu tego urządzenia,
ponownie działało bez zarzutu, umożliwiając pobranie informacji o~odebranej liczbie pakietów z~węzła dystalnego.
Sugeruje to raczej problemy ze stosem i~ewentualnym wyciekiem pamięci uniemożliwiające poprawne funkcjonowanie sprzętu.
Jednym z~kroków do dalszych analiz jest weryfikacja kodu i~jego działania w czasie rzeczywistym. Narzędzia jakie debugger
czy valgrind\footnote{\url{https://valgrind.org/}} byłyby pierwszym krokiem ku weryfikacji zagadnienia.

Ostatnią, aczkolwiek najistotniejszą zbadaną zależnością, jest wpływ dystansu na \gls{PER}. Zgodnie ze zdefiniowaną
hipotezą w~punkcie~\ref{subsec:per_experiment} i~opisywaną później metodologią, oczekiwano dominującą rolę otaczającego
środowiska radiowego w~przeprowadzanym eksperymencie.










% wyszukanie granicy na której następuje przełamanie trendu i normalne funkcjonowanie uC

Poza wymienionymi już możliwościami dalszego rozwoju badań, sugeruje się kolejne, następujące. Uwzględniając fakt utraty pakietów
ze względu na krótki interwał odpytywań, oczywistym jest wyszukanie punktu przełamania, w którym omawiana zależność nie występuje.
Celem by było wyznaczenie minimalnego interwału, który nie powoduje utraty pakietów na niewielkich odległościach w~kontrolowanych
warunkach.
