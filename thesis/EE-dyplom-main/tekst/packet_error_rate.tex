\section{Packet Error Rate}

Celem niniejszego podrozdziału jest omówienie przeprowadzonego eksperymentu \gls{PER}. Omówiona zostanie
metodologia badań. Zdefiniowany zostanie termin \textit{pakietu}, który stanowi podstawę dla
doświadczenia.

Wychodząc z definicji, prezentuje się właściwy wzór matematyczny, definiujący badany problem. Równanie
stanowi podstawę dla eksperymentu. Jego zrozumienie pozwoliło zaprojektowanie właściwego doświadczenia
jak i~przygotowanie kompletnego stosu technologicznego niezbędnego do jego przeprowadzenia.

Ostatecznym efektem przeprowadzonego eksperymentu jest przedstawienie zebranych danych pod postacią
wykresów. Prezentują one badane cechy zmienne zaprezentowane w sekcji opisu metodologii. Końcowym
krokiem jest wyciągnięcie wniosków z zebranych danych.
 
%%%%%%%%%%%%%%%%%%%%%%%%%%%%%%%%%%%%%%%%%%%%%%%%%%%%%%%%%%%%%%%%%%%%%%%%%%%%%%%%
%% SUBSECTION: Zależność PER względem odległości między węzłami
%%%%%%%%%%%%%%%%%%%%%%%%%%%%%%%%%%%%%%%%%%%%%%%%%%%%%%%%%%%%%%%%%%%%%%%%%%%%%%%%
\subsection{Metodologia badania}
\subsubsection{Definicja pakietu}
Przed przystąpieniem do badań należy precyzyjnie zdefiniować wykorzystywaną terminologię.

Pakietem nazywamy pojedynczą porcję danych przetwarzaną na poziomie warstwy sieciowej modelu ISO OSI \cite{sa_tcpip_nodate}.
Warstwa ta umożliwia routing, adresowanie logiczne oraz przetwarzanie i~dostarczanie pakietów.

\begin{figure}[!ht]
	\centering \includegraphics[width=0.618\linewidth]{tcp_ip_szkola_programowania_naglowki_stopki.png} 
	\caption{Model ISO OSI wraz i odpowiadające mu nazwy porcji danych. Źródło: \cite{sa_tcpip_nodate}}
	\label{rys:iso_osi_model_nazwy_grup_danych}
\end{figure}

\gls{BLE} wprowadza własną nomenklaturę dla poszczególnych warstw sieciowych. Jest to o tyle istotne, iż standard ten
nie zapewnia odpowiadających modelowi ISO OSI warstw jeden-do-jednego. Część z tych warstw jest agregowanych w~zespół 
protokołów wyższych warstw - Rysunek~\ref{rys:agregacja_protokolow_ble}. 

Bazując na definicji modelu OSI oraz stosie BLE, pakietem można nazwać wiadomości będące możliwie blisko
warstwy \textit{\gls{LL}}. Podobną definicję prezentuje dokumentacja ST:
\enquote{Pakiet to pojedyncza oznaczona wiadomość wysłana przez jedno i~odebrana przez 
co najmniej jedno urządzenie.}\footnote{Tłumaczenie własne}~\cite{stmicroelectronics_pm0271_2021}

\begin{figure}[!ht]
	\centering \includegraphics[width=0.618\linewidth]{mathworks_iso_osi_ble_stack.png} 
	\caption{Zestawienie stosu BLE i modelu ISO OSI. Źródło: \cite{noauthor_bluetooth_nodate}}
	\label{rys:agregacja_protokolow_ble}
\end{figure}

BLE Mesh dodatkowo wprowadza własne dodatkowe warstwy komunikacji, biorąc za podstawę stos BLE~\cite{mesh_working_group_mesh_2019}.
Każda z wymienionych warstw jest hermetyzowana za pośrednictwem dostarczanego przez producenta \gls{API}.
Uwzględniając zamknięcie middleware'u i dwu-procesorową architekturę mikrokontrolera STM32WB55, oznacza 
to brak możliwości bezpośredniego nasłuchiwania pakietów w~warstwie \gls{LL}.

Uwzględniając powyższe czynniki, \textit{pakietem} dla BLE Mesh nazywana będzie wiadomość najbliższa warstwie \gls{LL}.
W przypadku stworzonego oprogramowania, oznacza to odbiór komunikatu odebranego jako zdarzenie zarejestrowane przez
koprocesor Cortex-M0, będący integralną częścią mikrokontrolera STM32WB55 odpowiadający za obsługę radia.

\subsubsection{Definicja Packet Error Rate}
Posiadając definicję pakietu, \gls{PER} możliwe staje się zdefiniowanie wzoru, a zarazem znaczenia
głównego celu badań.

PER jest miarą ilości błędnych pakietów w proporcji do wszystkich wysłanych pakietów, zgodnie ze wzorem:

\begin{equation}
\label{per_equation}
PER = \frac{s - r}{s} \cdot 100\%
\end{equation}

gdzie:

\begin{description}
\item[s] is ilość wysłanych pakietów
\item[r] is ilość odebranych pakietów
\item[s-r] - ilość niepoprawnych/błędnych pakietów
\end{description}

Powyższy wzór stanowi podstawę eksperymentu pozwalającego wyznaczyć jakość łącza w zależności od
wybranych parametrów zmiennych.

\subsubsection{Procedura badawcza}

Procedurę badawczą skonstruowano bazując na wzorze~\ref{per_equation}. Niezbędnym mechanizmem, o które oparte
jest doświadczenie, to zliczanie ilości pakietów. Zliczanie dotyczy zarówno węzła bliższego 
jak i~również węzła dalszego odbierającego wysyłane komunikaty. W przypadku drugiego elementu 
opracowano mechanizm odczytywania licznika zmian badanej wartości, patrz: \ref{prep:uc-software}.

Całość doświadczenia przeprowadzono z wykorzystaniem oprogramowania PC - \ref{prep:pc-software}. Oprogramowanie
zapewnia dwie główne funkcjonalności: wysyłanie komunikatów przy określonej częstości przez zadaną ilość czasu;
odczytywanie wartości licznika węzła dalszego.

Wysyłanym komunikatem jest polecenie zmiany stanu modelu
\textit{Generic OnOff}. Wybrano ten standardowy element stosu BLE Mesh ze względu na jego uniwersalność.
Nie ogranicza się on jedynie do wybranej platformy czy własnościowego modelu. Model ten jest zdefiniowany
przez Bluetooth SIG przez co jest niezależny od producenta. Czyni to eksperyment powtarzalny,
niezależnie od mikrokontrolera.

Odczytywanie stanu licznika wymagało wykorzystania autorskiego rozwiązania oparte o własnościowy model Mesh
ST. Nie wpływa to jednak na ostateczny rezultat badań, gdyż stworzone polecenia wykorzystywane jest
tylko do odczytu i~wysłania wartości licznika węzła dalszego do węzła bliższego i~komputera osobistego kontrolującego
przepływ doświadczenia. Z każdą kolejną próbą badania PER ten licznik jest automatycznie zerowany,
przez co sesja zliczeń zawsze rozpoczyna się od zera.

Eksperyment wyznaczający PER oparto o następujące czynniki zmienne:
\begin{itemize}
\item środowisko: teren leśny, teren zurbanizowany
\item interwał zapytań
\item dystans pomiędzy węzłami
\item ilość węzłów składających się na sieć Mesh
\end{itemize}

Wyznaczanie PER odbyło się w dwóch różnych środowiskach. Jednym z głównych hipotez jest znaczący wpływ
środowiska na jakość transmisji danych. Czynniki takie jak temperatura, wilgotność, rodzaj gleby czy
tło radiowe może mieć wpływ na komunikację pomiędzy węzłami. Założono, iż tło radiowe może mieć
największy wpływ ja transmisję danych. Stąd dobrano możliwie skrajne miejsca do badań oceniając
to jako najistotniejszy czynnik:
\begin{itemize}
\item Kampinowski Park Narodowy (lokalizacja: parking Roztoka) - jako teren leśny oddalony od ośrodka miejskiego
ze względnie niewielkim tłem radiowym. Pogoda: pochmurnie, wilgotno, temperatura poniżej 20$^{\circ}$C.
\item Park Pola Mokotowskie - jako teren zurbanizowany charakteryzujący się bogatym tłem radiowym działającym
w pasmach \gls{ISM}. Pogoda: słonecznie, temperatura ok. 20$^{\circ}$C.
\end{itemize}

Kolejnym badanym czynnikiem jest interwał zapytań. Parametr ten został wybrany ze względu na obserwowane
problemy z komunikacją podczas etapu tworzenia oprogramowania. Parametry dobrano w takim stopniu, by ów problem
ukazać. Obrano następujące interwały:
\begin{itemize}
\item 100ms
\item 500ms
\item 800ms
\item 1300ms
\item 2100ms
\end{itemize}

Interesującym parametrem dla bezprzewodowej transmisji danych jest zasięg. Stąd też, jednym z badanych czynników jest
określenie jakości PER w zależności od odległości - Rysunek~\ref{rys:two_nodes_setup}.
\begin{itemize}
\item 1,5m
\item 3,0m
\item 5,0m
\item 8,0m
\item 13,0m
\item 16,0m
\item 21,0m
\end{itemize}

\begin{figure}[!ht]
	\centering \includegraphics[width=0.618\linewidth]{per_two_nodes.pdf} 
	\caption{Dystans pomiędzy węzłami dla sieci dwóch mikrokontrolerów}
	\label{rys:two_nodes_setup}
\end{figure}

Wyżej wymienione odległości stosowano również w przypadku kolejnego badanego parametru, tj. ilości węzłów
składających się na sieć BLE. W celu łatwej identyfikacji węzłów, wprowadza się następujące nazewnictwo:
\begin{itemize}
	\item węzeł bliższy (ang./łac. \textit{proximal node}/\textit{nodus proximalis}) - węzeł będący połączony bezpośrednio
	ze stacją akwizycji danych i kontroli przepływu eksperymentu.
	\item węzeł środkowy (ang./łac. \textit{intermedial node}/\textit{nodus [inter]medius}) - węzeł działający w trybie
	przekaźnika (terminologia Mesh: \textit{Relay}). Węzeł ten nie uczestniczy bezpośrednio w badaniach tj. nie są
	z niego odczytywane jakiekolwiek dane.
	\item węzeł dalszy (ang./łac. \textit{distal node}/\textit{nodus distalis}) - węzeł zliczający ilość odebranych danych \textit{r},
	udostępniający jednocześnie usługę umożliwiającą odczyt tych wartości tak jak opisano to w podrozdziale \ref{prep:uc-software}.
\end{itemize}

Postanowiono o równoodległym rozstawieniu węzłów - Rysunek~\ref{rys:three_nodes_setup}. Dla sieci 3 węzłów,
maksymalna odległość dzieląca węzeł bliższy od węzła dalszego to 42m. Protokół badawczy zawiera informację 
tylko o odległości w~rozumieniu równoodległego rozstawienia węzłów. Odległość pomiędzy elementami 
jest oczywistą operacją arytmetyczną.

\begin{figure}[!ht]
	\centering \includegraphics[width=0.99\linewidth]{per_three_nodes.pdf} 
	\caption{Dystans pomiędzy węzłami dla sieci trzech mikrokontrolerów}
	\label{rys:three_nodes_setup}
\end{figure}

Odległość pomiędzy węzłami mierzona jest z użyciem taśmy mierniczej z podziałką 1mm. Tolerancję pomiarów należy
przyjąć jako najdłuższy wymiar zestawu uruchomieniowego P-NUCLEO-WB55 - 70mm \cite{stmicroelectronics_um2435_2019}.


Parametry transmisji danych nie ulegały zmianie podczas przeprowadzanych doświadczeń. Poniższe wartości należy
przyjąć za stałe:
\begin{itemize}
\item Szybkość transmisji: 2Mbps
\item Moc transmisji danych: 0dBm
\end{itemize}

%%%%%%%%%%%%%%%%%%%%%%%%%%%%%%%%%%%%%%%%%%%%%%%%%%%%%%%%%%%%%%%%%%%%%%%%%%%%%%%%
%% SUBSECTION: Zależność \gls{PER} względem częstości zapytań
%%%%%%%%%%%%%%%%%%%%%%%%%%%%%%%%%%%%%%%%%%%%%%%%%%%%%%%%%%%%%%%%%%%%%%%%%%%%%%%%
\subsection{Zależność PER względem częstości zapytań}

\begin{figure}[!htb]
	\centering \includegraphics[width=0.618\linewidth]{per_to_distance_under_100ms.png}
	\caption{Zależność \gls{PER} od dystansu dla zapytań o częstości $\leqslant$ 100ms dla różnej liczby węzłów}
	\label{rys:per_to_distance_under_100ms}
\end{figure}

\lipsum[1-3]
\begin{figure}[!htb]
	\centering \includegraphics[width=0.618\linewidth]{per_to_distance_under_100ms_different_envs.png} 
	\caption{Zależność \gls{PER} od dystansu dla zapytań o częstości $\leqslant$ 100ms w wybranych środowiskach bez rozróżnienia na liczbę węzłów}
	\label{rys:per_to_distance_under_100ms_different_envs}
\end{figure}

\lipsum[1-3]
\begin{figure}[!htb]
	\centering \includegraphics[width=0.99\linewidth]{per_to_distance_under_100ms_different_envs_and_nodes.png}
	\caption{Zależność \gls{PER} od dystansu dla zapytań o częstości $\leqslant$ 100ms w wybranych środowiskach i liczbę badanych węzłów}
	\label{rys:per_to_distance_under_100ms_different_envs_and_nodes}
\end{figure}

\lipsum[1-3]
\begin{figure}[!htb]
	\centering \includegraphics[width=0.99\linewidth]{per_to_distance_over_100ms_different_envs_different_ping_interval.png} 
	\caption{Zależność \gls{PER} od dystansu dla zapytań o częstości >100ms w wybranych środowiskach}
	\label{rys:per_to_distance_over_100ms_different_envs_different_ping_interval}
\end{figure}
\lipsum[1-3]


%%%%%%%%%%%%%%%%%%%%%%%%%%%%%%%%%%%%%%%%%%%%%%%%%%%%%%%%%%%%%%%%%%%%%%%%%%%%%%%%
%% SUBSECTION: Zależność \gls{PER} względem odległości między węzłami
%%%%%%%%%%%%%%%%%%%%%%%%%%%%%%%%%%%%%%%%%%%%%%%%%%%%%%%%%%%%%%%%%%%%%%%%%%%%%%%%
\subsection{Zależność PER względem odległości między węzłami}

\begin{figure}[!htb]
	\centering \includegraphics[width=0.618\linewidth]{per_to_distance_over_100ms.png}
	\caption{Zależność \gls{PER} od dystansu dla zapytań o częstości >100ms dla różnej liczby węzłów}
	\label{rys:per_to_distance_over_100ms}
\end{figure}

\begin{figure}[!htb]
	\centering \includegraphics[width=0.618\linewidth]{per_to_distance_over_100ms_different_envs.png} 
	\caption{Zależność \gls{PER} od dystansu dla zapytań o częstości >100ms w wybranych środowiskach bez rozróżnienia na liczbę węzłów}
	\label{rys:per_to_distance_over_100ms_different_envs}
\end{figure}

\begin{figure}[!htb]
	\centering \includegraphics[width=0.99\linewidth]{per_to_distance_over_100ms_different_envs_and_nodes.png}
	\caption{Zależność \gls{PER} od dystansu dla zapytań o częstości >100ms w wybranych środowiskach i liczbę badanych węzłów}
	\label{rys:per_to_distance_over_100ms_different_envs_and_nodes}
\end{figure}


