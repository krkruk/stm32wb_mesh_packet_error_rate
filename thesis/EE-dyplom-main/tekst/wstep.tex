Od czasów powstania pierwszych procesorów, naukowcy i~inżynierowie łączyli
je z~dostępnymi elementami dyskretnymi tworząc urządzenia spełniające
określone potrzeby. Z~każdym przełomem w miniaturyzacji komponentów
cechą wspólną jest tworzenie zespołów współpracujących ze sobą elementów.
Od elektroniki analogowej, do obecnych możliwości elektroniki cyfrowej,
wymiana informacji, wysyłanie i odbiór sygnałów, jest kluczem do 
stworzenia zaawansowanego urządzenia.

Intuicyjnie, termin sygnał przynosi na myśl pojęcie nośnika informacji, czy wymiany tejże informacji.
Naturalnie łączy się to słowo z~wielkościami fizycznymi a nawet namacalnym. Docelowo, pragniemy wysłać
pewną treść, co też wiąże się z~rozdrobnieniem tak abstrakcyjnej koncepcji na możliwe małe fragmenty,
które następnie można przesłać dalej. Chcąc opisać termin nie tylko jakościowo, ale też ilościowo,
kształtuje się taką ideę do postaci \textit{modelu matematycznego}. Sygnałem więc nazwać można pewną funkcję
czasu opisującą zjawisko przesyłu tej informacji \cite{szabatin_podstawy_2007}.

Wraz z~rozwojem techniki, opracowano właściwe technologie i~ustalono kontrakty definiujące kodowanie
owego sygnału. Nie mniej istotną cechą jest wybór zjawiska fizycznego, które tę wiadomość ma przesyłać.
Od tego zależy sposób konstrukcji urządzeń odpowiadających za przesył danych. Inne narzędzia
należy wykorzystać do komunikacji z użyciem znaków dymnych, a zupełnie innych do kontaktu
z~łazikiem marsjańskim oddalonym o 20 minut świetlnych od Ziemi. Przesył sygnału nie jest cechą
wyłącznie innowacyjności ludzkich dzieł. Natura w~wyniku ewolucji opracowała szereg
czynników umożliwiający przesyłanie i odbiór informacji -- powonienie, smak, transport aktywny jonów
w postaci pompy sodowo-potasowej będącą podstawą dla transmisji sygnałów w komórkach nerwowych itd.
Sygnał i~możliwość manipulacji zdaje się być podstawą funkcjonowania nie tyle cywilizacji ludzkiej,
co życia samego w sobie.

Technologicznie, sygnał przesyłać można z użyciem szeregu zjawisk fizycznych, najczęściej powiązanych 
ze zjawiskami mechanicznymi oraz elektrycznymi. Skupiając się na tych ostatnich, sygnał generuje się
bazując na pojęciach napięcia, prądu, częstotliwości, czy ogólnie fal elektromagnetycznych.
Telegraf, czy jego współczesne wersje w postaci telefonu czy Internetu, manipulują tą falą z~użyciem
technologii celem wysłania informacji z punktu A~ do punktu~B. Sposób w~jaki wpływamy
na otaczające środowisko, by wysłać sygnał, modelowo nazywane jest warstwą fizyczną~\cite{sa_tcpip_nodate}.

Kolejnym krokiem jest ustalenie pewnego kontraktu pomiędzy wspomnianą warstwą fizyczną, a~potencjalnymi
warstwami wyższymi. Wymagany jest sposób według którego można zinterpretować ilościowy udział
zjawiska fizycznego. W przypadku elektroniki cyfrowej bazującej na \textit{TTL}\footnote{tutaj: Transistor-Transistor Logic},
informacja kodowana jest z użyciem zmian napięcia, gdzie sygnałem niskim (czyli logiczne \enquote{0}) nazywamy 
napięcie w zakresie 0V do 0.8V, a stan wysoki (czyli logiczne \enquote{1}) od 2.4 do maksymalnego napięcia 5V.
Nowoczesne systemy ten prosty przykład znacząco modyfikują, by przesyłać więcej danych w~krótszym czasie,
najlepiej na dalszy dystans z~minimalizacją zużycia energii. Sposób w~jaki jest to zorganizowane,
można przyrównywać właśnie do modelu \gls{OSI} stanowiącego pewien schemat rozumowania przesyłu sygnału
cyfrowego w~sieci.

Poprzez analogię do wyżej wymienionych zjawisk, niniejsza praca podejmuje się badania właściwości
jednej z~technologii wykorzystujących podstawy fizyczne do bezprzewodowego przesyłu informacji. 
Technologią tą jest Bluetooth~5 będąca rozwinięciem standardu funkcjonującego od ponad dwudziestu lat.
Wykorzystywana z~powodzeniem w elektronice użytkowej wraz z nową wersją wprowadza szereg usprawnień,
które umożlwiją zastosowanie aplikacjach IoT.

\gls{IoT} jest systemem złożonym z elementów (\textit{rzeczy}) połączonych ze sobą w sieć, umożliwiającą
wysyłanie, przesyłanie i~przetwarzanie danych pomiędzy poszczególnymi węzłami. Zagadnienie łączy
we wspólną całość mikrokontrolery, czy jednopłytkowe komputery oparte o układ scalony typu \gls{SoC},
wraz z~większą siecią nie wykluczając Internetu. Zagadnienie tym samym wiąże nie tylko opis protokołów
transmisji danych, ale wymaga skupienia również na sprzęcie stanowiąc
kompletny przepis na docelowe rozwiązanie techniczne~\cite{mcewen_designing_2013}.

Internet Rzeczy koncentruje się na szerokim aspekcie udostępniania urządzeń o małej ilości obliczeniowej
do różnych sieci. Obejmuje to definicję metod transmisji danych, właściwe protokoły czy typy
transmitowanych danych, ale również kwestię bezpieczeństwa -- szyfrowanie. W swej myśli, IoT
jest systemem powszechnym, tudzież urządzenia korzystającej z~tej koncepcji są powszechne.
Przykładami zastosowań są między innymi transport (np. znaki drogowe, zmieniające ograniczenia
prędkości w~zależności od warunków pogodowych), sprzęt medyczny (np. mobilne rejestratory EKG -- monitorowanie
metodą Holtera) i konsumencki (np. inteligentne zegarki monitorujące tętno), przemysł, sieci
sensoryczne i~wiele innych. Oczekiwaną najczęściej cechą takich systemów jest energooszczędność,
gdyż podłączone urządzenia często zasilane są bateryjnie. IoT obejmuje więc szerokie spektrum
dziedzin inżynierii oraz codziennego życia.

Celem niniejszej pracy jest zapoznanie się i~zbadanie parametrów nowo wprowadzonego standardu począwszy
od Bluetooth~5 -- Bluetooth Mesh. Technologia ta obecna od niedawna na rynku, wprowadza istotne
zmiany, w~tym właśnie tworzenie sieci urządzeń, które mogą ze sobą się wzajemnie komunikować w~sposób
zdefiniowany przez standard. Badane są zużycie energii układów wspierających Mesh i~Bluetooth
Low Energy oraz jakość łącza transmisji danych poprzez parametr \gls{PER}.

Rozdział~\ref{ch:radio-telecom} zapoznaje czytelnika z powszechnymi na rynku rozwiązaniami, protokołami:
\begin{itemize}
\item ZigBee
\item Thread
\item Bluetooth~5 z naciskiem na standard Bluetooth Low Energy i~Mesh
\end{itemize}

Dla każdej wyżej wymienionej technologii, wprowadza się podstawowe pojęcia z nią związane. Rozważania
poparte są o~specyfikacje opracowane przez właściwe organizacje nadzorujące prace nad każdym ze
standardów. Rozwiązania powstają wraz z udziałem firm trzecich, technologicznych, czy producentów
elektroniki użytkowej, uwzględniając wspólną wizję i~realne zapotrzebowania. Przedstawiając
stos technologiczny, niniejsza praca odwołuje się do modelu \gls{OSI}, jako referencyjnego umożliwiając
umieszczenie poszczególnych terminów i~nazw we wspólnym standardzie.

Wprowadzając powyższe technologie, uwaga koncentrowana jest na aspekcie konfigurowania sieci
składających się z wielu elementów. Tym samym, niezbędne jest wprowadzenie terminologii z~jakimi
wiąże się każda ze specyfikacji. Nie mniej istotne są możliwe topologie czy sposób przesyłu
pakietów danych pomiędzy węzłami -- tzw. routing.

Po teoretycznym wprowadzeniu do poszczególnych protokołów, następuje przegląd powstałej dotąd
literatury. Oczywiste staje się, iż badanie właściwości sieci, ich skalowalności, przepustowości
i~innych parametrów cieszy się ogromnym zainteresowaniem zarówno naukowców jak i~poszczególnych
firm produkujące właściwą do ich obsługi elektronikę. Docelowo, zestawiane są parametry poszczególnych
protokołów, pozwalając czytelnikowi na samodzielną kontemplację.

Wymieniony rozdział zakończony jest wprowadzeniem do dwóch prowadzonych i~opisywanych eksperymentów:
badania zużycia energii urządzeń BLE i BT Mesh oraz badania jakości sieci mierzonej wartością
Packet Error Rate. Wprowadza się niezbędną terminologię oraz matematyczny opis wymienionych zjawisk
i~parametrów celem zastosowania ich w~dalszej analizie zebranych danych. Definiowane są cele
badawcze oraz sprawdzane hipotezy. Dla eksperymentu mierzącego zużycie energii, sprawdza się
pobór prądu elektrycznego w~czasie. Zebrane wyniki porównywane są następnie z~wartościami
symulowanymi z~użyciem narzędzi dostarczanych przez producenta mikrokontrolera.
Packet Error Rate pozwala na określenie jakości i~stabilności połączenia. Głównymi hipotezami
badawczymi są: wpływ dystansu na PER -- wraz ze wzrostem dystansu oczekuje się wzrostu PER;
wpływ środowiska na PER -- zakłada się, teren o~bogatym tle radiowym (tutaj: teren zurbanizowany)
negatywnie wpłynie na badany współczynnik, uzyskując większą utratę przesyłanych pakietów,
niż w~przypadku o~uboższym tle radiowym (tutaj: teren miejski).

Główną treścią celem pracy jest rozdział~\ref{ch:experiment}, wprowadzający czytelnika w~aspekty praktyczne opisywanych
rozważań. Mając na uwadze wymienione wcześniej planowane doświadczenia, należy przygotować właściwy tor
pomiarowy, by zebrać dane, celem dalszej analizy. Do celów pracy, wykorzystano
gotowe komercyjnie dostępne zestawy uruchomieniowe \textit{P-NUCLEO-WB} oraz pomiarowy
\textit{X-NUCLEO-LPM01A}. Opierając się na wybranym sprzęcie, wprowadza w~szczegóły
tworzenia niezbędnych urządzeń wspomagających i~oprogramowania koniecznego
do przeprowadzenia badań.

Przedstawiana jest metodologia poszczególnych eksperymentów. Określone są warunki początkowe,
parametry techniczne badanych właściwości. Prezentuje się również metodologię oraz
nawiązuje do właściwych, wcześniej wymienionych, wzorów i~zależności, o które oparto
zebrane dane. Ostatecznie, prezentowane są wykresy nakreślające cechy zebranych pomiarów.
Wykresy, oprócz rzeczywistych danych, prezentują linie aproksymacyjne, celem określenia
tendencji wzrostów/spadków PER.

Finalnie, przedstawiane są wnioski zarówno z~przygotowań do przeprowadzenia eksperymentów
jak i~przede wszystkim wyciągane są należyte obserwacje i~ich konsekwencje, prezentując równocześnie 
dalsze kierunki rozwoju opisywanej tej pracy analizy.

