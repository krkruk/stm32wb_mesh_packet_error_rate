% !TEX program = xelatex
% !TeX encoding = utf8
% !TeX spellcheck = pl-PL

%%%%%%%%%%%%%%%%%%%%%%%%%%%%%%%%%%%%%%%%%%%%%%%%%%%%%%%%%%%%%%%%%%%%%%%%%%%
% Wybierz rodzaj pracy dyplomowej oraz wydział
% Pick thesis type and faculty
%%%%%%%%%%%%%%%%%%%%%%%%%%%%%%%%%%%%%%%%%%%%%%%%%%%%%%%%%%%%%%%%%%%%%%%%%%%
\documentclass[thesis=mgr,faculty=ee]{EE-dyplom} 

% thesis=[inz|mgr|bsc|msc]
%  * inz - praca inżynierska
%  * mgr - praca magisterska
%  * bsc - bachelor thesis
%  * msc - master thesis

% Skróty nazw wydziałów zgodne z domenami internetowymi
% Abbreviations of Faculties according to Internet subdomains
% faculty=[
%	arch,
%	gik,
%	ee,
%	wip
%	]

%%%%%%%%%%%%%%%%%%%%%%%%%%%%%%%%%%%%%%%%%%%%%%%%%%%%%%%%%%%%%%%%%%%%%%%%%%%
% Konfiguracja - do personalizacji
% Configuration - to be personalized
%%%%%%%%%%%%%%%%%%%%%%%%%%%%%%%%%%%%%%%%%%%%%%%%%%%%%%%%%%%%%%%%%%%%%%%%%%%
\instytut{Instytut Elektrotechniki Teoretycznej i Systemów Informacyjno-Pomiarowych}
\kierunek{Informatyka Stosowana}
\specjalnosc{Inżynieria Oprogramowania}
\title{Nowe możliwości Bluetooth 5 w aplikacjach Internetu Rzeczy}
\engtitle{New Bluetooth 5 features in Internet of Things applications}
\album{301140}
\author{inż. Krzysztof Paweł Kruk}
\promotor{dr inż. Łukasz Makowski}
\date{2022}
\longdate{2022-05-10}

%\grantlicense{TRUE} % [TRUE|FALSE]

%%%%%%%%%%%%%%%%%%%%%%%%%%%%%%%%%%%%%%%%%%%%%%%%%%%%%%%%%%%%%%%%%%%%%%%%%%%
% Streszczenie pracy i abstract.
% In case of thesis in English swap the order - English version goes first.
%%%%%%%%%%%%%%%%%%%%%%%%%%%%%%%%%%%%%%%%%%%%%%%%%%%%%%%%%%%%%%%%%%%%%%%%%%%
\streszczeniepracy{
Jednym z dostępnych na rynku standardów komunikacji bezprzewodowej bliskiego
zasięgo jest Bluetooth. Obecny od ponad dwudziestu lat wśród
elektroniki użytkowej zdobył ugruntowaną pozycję i~popularność. Bluetooth~5
rozszerza znane wcześniej funkcjonalności, zwiększa przepustowość
i zasięg przy jednoczesnej redukcji zużycia energii oraz wprowadza nowy 
standard komunikacji sieci Mesh -- nowy protokół komunikacji między-węzłowej.

Niniejsza praca wprowadza w~świat łączności bezprzewodowej
opisując jedne z najpopularniejszych obecnie standardów. Wprowadza
się terminologię dla wybranych stosów technologicznych oraz zestawia z~modelem
referencyjnym ISO OSI. Wśród protokołów opartych o IEEE 802.15.4 wymienia się 
specyfikację ZigBee oraz Thread. Przedstawiany jest opis zmian Bluetooth 
od wersji 5.0. Ostatecznie, zestawia się wybrane cechy każdego z~wymienionych
standardów.

Przegląd literatury wskazuje na szerokie zainteresowanie nauki dziedziną
łączności bezprzewodowej. Szereg arykułów prezentuje empirycznie
zebrane cechy poszczególnych standardów takie jak opóźnienia transmisji
czy maksymalna przepustowość.

Doświadczalnym aspektem pracy jest pomiar zużycia energii i~pomiar Packet
Error Rate w zależności od wzajemnej odległości węzłów w~skonfigurowanych w~sieć 
Bluetooth Mesh i~częstotliwości wysłanych komunikatów. Stawiana jest główna 
hipoteza badawcza, która następnie
jest empirycznie weryfikowana. W~tym celu konstruowany jest tor
badawczy oparty o zestaw uruchomieniowy \textit{P-NUCLEO-WB55} oraz
autorskie oprogramowanie. Przedstawiana jest metodologia poszczególnych
badań oraz analiza potencjalnych źródeł błędów.

Ostatecznie, prezentowane są wyniki eksperymentów. Zestawiane są
wartości zużycia energii dla Bluetooth Low Energy i~Mesh.
Analizowany są przypadki badań Packet Error Rate, które następnie
podzielone są na dwie różne klasyfikacje które zostają
właściwie opisane.

Na koniec, praca podsumowuje dokonane prace i~wyciąga konstruktywne
wnioski.


} % koniec streszczenia

\slowakluczowe{Bluetooth, Bluetooth Low Energy, Bluetooth Mesh, Packet Error Rate, Zużycie energii}

\thesisabstract{
Bluetooth is a widely-available short-range wireless technology. The technology
has been present for more than twenty years in the market gaining undisputable 
popularity among consumer electronics. Bluetooth~5 pushes the limits further,
providing a new set of features, increasing throughput and range while simulanously
reduces energy requirements. Most importantly, the new standard introduces
Bluetooth Mesh -- a new wireless inter-node protocol.

This thesis introduces into a realm of wireless technologies. It describes
a range of wireless standards known to be the most popular. It defines
the terminologies for a given set of technology stacks and compares them
with a conceptual model -- \gls{OSI}. ZigBee and Thread represent a family
of IEEE 802.15.4-based stacks. The thesis describes a~set of Bluetooth~5 changes.
Finally, it juxtaposes the mentioned standards with each other.

The literature review suggests thats wireless communication is a common and
popular field of study. There are many papers that test parameters of
each of said stacks. These parameters include but not limited to -- latency, throughput
energy consumption and other.

The practcal apsect of the thesis embraces two experiments -- energy consumption
and Packet Error Rate in Bluetooth Mesh network. The latter experiment includes 
dependent factors such as distance between nodes or ping frequency.
The thesis states the main test hypothesis and tries to empirically verify it.
All the experiments are based on an off-the-shelf development board -- \textit{P-NUCLEO-WB55}.
The board is programmed and operated by custom software specifically designed
to run the mentioned experiment. The thesis includes a methodology description
as well as an analysis of possible error root causes.

Eventually, the thesis presents results of the abovementioned experiments.
It presents Bluetooth Low Energy and Mesh energy consumption charts
and compares them. It analyzes Packet Error Rate results by dividing it
into two distinct categories and presenting charts accordingly.

Finally, the thesis summarizes all experiments and draws conclusions.

} % end of abstract

\thesiskeywords{Bluetooth, Bluetooth Low Energy, Bluetooth Mesh, Packet Error Rate, Energy consumption}

%%%%%%%%%%%%%%%%%%%%%%%%%%%%%%%%%%%%%%%%%%%%%%%%%%%%%%%%%%%%%%%%%%%%%%%%%%%
% Tu zaczyna się dokument
% Here is the beginning of the document
%%%%%%%%%%%%%%%%%%%%%%%%%%%%%%%%%%%%%%%%%%%%%%%%%%%%%%%%%%%%%%%%%%%%%%%%%%%
\begin{document}
    % Strony nagłówkowe
    % Headers
    \frontpages

    % Właściwa treść jest w pliku tekst/main.tex
    % Real contents is in tekst/main.tex
    \chapter{Wstęp}
Od czasów powstania pierwszych procesorów, naukowcy i~inżynierowie łączyli
je z~dostępnymi elementami dyskretnymi tworząc urządzenia spełniające
określone potrzeby. Z~każdym przełomem w miniaturyzacji komponentów
cechą wspólną jest tworzenie zespołów współpracujących ze sobą elementów.
Od elektroniki analogowej, do obecnych możliwości elektroniki cyfrowej,
wymiana informacji, wysyłanie i odbiór sygnałów, jest kluczem do 
stworzenia zaawansowanego urządzenia.

Intuicyjnie, termin sygnał przynosi na myśl pojęcie nośnika informacji, czy wymiany tejże informacji.
Naturalnie łączy się to słowo z~wielkościami fizycznymi a nawet namacalnym. Docelowo, pragniemy wysłać
pewną treść, co też wiąże się z~rozdrobnieniem tak abstrakcyjnej koncepcji na możliwe małe fragmenty,
które następnie można przesłać dalej. Chcąc opisać termin nie tylko jakościowo, ale też ilościowo,
kształtuje się taką ideę do postaci \textit{modelu matematycznego}. Sygnałem więc nazwać można pewną funkcję
czasu opisującą zjawisko przesyłu tej informacji \cite{szabatin_podstawy_2007}.

Wraz z~rozwojem techniki, opracowano właściwe technologie i~ustalono kontrakty definiujące kodowanie
owego sygnału. Nie mniej istotną cechą jest wybór zjawiska fizycznego, które tę wiadomość ma przesyłać.
Od tego zależy sposób konstrukcji urządzeń odpowiadających za przesył danych. Inne narzędzia
należy wykorzystać do komunikacji z użyciem znaków dymnych, a zupełnie innych do kontaktu
z~łazikiem marsjańskim oddalonym o 20 minut świetlnych od Ziemi. Przesył sygnału nie jest cechą
wyłącznie innowacyjności ludzkich dzieł. Natura w~wyniku ewolucji opracowała szereg
czynników umożliwiający przesyłanie i odbiór informacji -- powonienie, smak, transport aktywny jonów
w postaci pompy sodowo-potasowej będącą podstawą dla transmisji sygnałów w komórkach nerwowych itd.
Sygnał i~możliwość manipulacji zdaje się być podstawą funkcjonowania nie tyle cywilizacji ludzkiej,
co życia samego w sobie.

Technologicznie, sygnał przesyłać można z użyciem szeregu zjawisk fizycznych, najczęściej powiązanych 
ze zjawiskami mechanicznymi oraz elektrycznymi. Skupiając się na tych ostatnich, sygnał generuje się
bazując na pojęciach napięcia, prądu, częstotliwości, czy ogólnie fal elektromagnetycznych.
Telegraf, czy jego współczesne wersje w postaci telefonu czy Internetu, manipulują tą falą z~użyciem
technologii celem wysłania informacji z punktu A~ do punktu~B. Sposób w~jaki wpływamy
na otaczające środowisko, by wysłać sygnał, modelowo nazywane jest warstwą fizyczną~\cite{sa_tcpip_nodate}.

Kolejnym krokiem jest ustalenie pewnego kontraktu pomiędzy wspomnianą warstwą fizyczną, a~potencjalnymi
warstwami wyższymi. Wymagany jest sposób według którego można zinterpretować ilościowy udział
zjawiska fizycznego. W przypadku elektroniki cyfrowej bazującej na \textit{TTL}\footnote{tutaj: Transistor-Transistor Logic},
informacja kodowana jest z użyciem zmian napięcia, gdzie sygnałem niskim (czyli logiczne \enquote{0}) nazywamy 
napięcie w zakresie 0V do 0.8V, a stan wysoki (czyli logiczne \enquote{1}) od 2.4 do maksymalnego napięcia 5V.
Nowoczesne systemy ten prosty przykład znacząco modyfikują, by przesyłać więcej danych w~krótszym czasie,
najlepiej na dalszy dystans z~minimalizacją zużycia energii. Sposób w~jaki jest to zorganizowane,
można przyrównywać właśnie do modelu \gls{OSI} stanowiącego pewien schemat rozumowania przesyłu sygnału
cyfrowego w~sieci.

Poprzez analogię do wyżej wymienionych zjawisk, niniejsza praca podejmuje się badania właściwości
jednej z~technologii wykorzystujących podstawy fizyczne do bezprzewodowego przesyłu informacji. 
Technologią tą jest Bluetooth~5 będąca rozwinięciem standardu funkcjonującego od ponad dwudziestu lat.
Wykorzystywana z~powodzeniem w elektronice użytkowej wraz z nową wersją wprowadza szereg usprawnień,
które umożlwiją zastosowanie aplikacjach IoT.

\gls{IoT} jest systemem złożonym z elementów (\textit{rzeczy}) połączonych ze sobą w sieć, umożliwiającą
wysyłanie, przesyłanie i~przetwarzanie danych pomiędzy poszczególnymi węzłami. Zagadnienie łączy
we wspólną całość mikrokontrolery, czy jednopłytkowe komputery oparte o układ scalony typu \gls{SoC},
wraz z~większą siecią nie wykluczając Internetu. Zagadnienie tym samym wiąże nie tylko opis protokołów
transmisji danych, ale wymaga skupienia również na sprzęcie stanowiąc
kompletny przepis na docelowe rozwiązanie techniczne~\cite{mcewen_designing_2013}.

Internet Rzeczy koncentruje się na szerokim aspekcie udostępniania urządzeń o małej ilości obliczeniowej
do różnych sieci. Obejmuje to definicję metod transmisji danych, właściwe protokoły czy typy
transmitowanych danych, ale również kwestię bezpieczeństwa -- szyfrowanie. W swej myśli, IoT
jest systemem powszechnym, tudzież urządzenia korzystającej z~tej koncepcji są powszechne.
Przykładami zastosowań są między innymi transport (np. znaki drogowe, zmieniające ograniczenia
prędkości w~zależności od warunków pogodowych), sprzęt medyczny (np. mobilne rejestratory EKG -- monitorowanie
metodą Holtera) i konsumencki (np. inteligentne zegarki monitorujące tętno), przemysł, sieci
sensoryczne i~wiele innych. Oczekiwaną najczęściej cechą takich systemów jest energooszczędność,
gdyż podłączone urządzenia często zasilane są bateryjnie. IoT obejmuje więc szerokie spektrum
dziedzin inżynierii oraz codziennego życia.

Celem niniejszej pracy jest zapoznanie się i~zbadanie parametrów nowo wprowadzonego standardu począwszy
od Bluetooth~5 -- Bluetooth Mesh. Technologia ta obecna od niedawna na rynku, wprowadza istotne
zmiany, w~tym właśnie tworzenie sieci urządzeń, które mogą ze sobą się wzajemnie komunikować w~sposób
zdefiniowany przez standard. Badane są zużycie energii układów wspierających Mesh i~Bluetooth
Low Energy oraz jakość łącza transmisji danych poprzez parametr \gls{PER}.

Rozdział~\ref{ch:radio-telecom} zapoznaje czytelnika z powszechnymi na rynku rozwiązaniami, protokołami:
\begin{itemize}
\item ZigBee
\item Thread
\item Bluetooth~5 z naciskiem na standard Bluetooth Low Energy i~Mesh
\end{itemize}

Dla każdej wyżej wymienionej technologii, wprowadza się podstawowe pojęcia z nią związane. Rozważania
poparte są o~specyfikacje opracowane przez właściwe organizacje nadzorujące prace nad każdym ze
standardów. Rozwiązania powstają wraz z udziałem firm trzecich, technologicznych, czy producentów
elektroniki użytkowej, uwzględniając wspólną wizję i~realne zapotrzebowania. Przedstawiając
stos technologiczny, niniejsza praca odwołuje się do modelu \gls{OSI}, jako referencyjnego umożliwiając
umieszczenie poszczególnych terminów i~nazw we wspólnym standardzie.

Wprowadzając powyższe technologie, uwaga koncentrowana jest na aspekcie konfigurowania sieci
składających się z wielu elementów. Tym samym, niezbędne jest wprowadzenie terminologii z~jakimi
wiąże się każda ze specyfikacji. Nie mniej istotne są możliwe topologie czy sposób przesyłu
pakietów danych pomiędzy węzłami -- tzw. routing.

Po teoretycznym wprowadzeniu do poszczególnych protokołów, następuje przegląd powstałej dotąd
literatury. Oczywiste staje się, iż badanie właściwości sieci, ich skalowalności, przepustowości
i~innych parametrów cieszy się ogromnym zainteresowaniem zarówno naukowców jak i~poszczególnych
firm produkujące właściwą do ich obsługi elektronikę. Docelowo, zestawiane są parametry poszczególnych
protokołów, pozwalając czytelnikowi na samodzielną kontemplację.

Wymieniony rozdział zakończony jest wprowadzeniem do dwóch prowadzonych i~opisywanych eksperymentów:
badania zużycia energii urządzeń BLE i BT Mesh oraz badania jakości sieci mierzonej wartością
Packet Error Rate. Wprowadza się niezbędną terminologię oraz matematyczny opis wymienionych zjawisk
i~parametrów celem zastosowania ich w~dalszej analizie zebranych danych. Definiowane są cele
badawcze oraz sprawdzane hipotezy. Dla eksperymentu mierzącego zużycie energii, sprawdza się
pobór prądu elektrycznego w~czasie. Zebrane wyniki porównywane są następnie z~wartościami
symulowanymi z~użyciem narzędzi dostarczanych przez producenta mikrokontrolera.
Packet Error Rate pozwala na określenie jakości i~stabilności połączenia. Głównymi hipotezami
badawczymi są: wpływ dystansu na PER -- wraz ze wzrostem dystansu oczekuje się wzrostu PER;
wpływ środowiska na PER -- zakłada się, teren o~bogatym tle radiowym (tutaj: teren zurbanizowany)
negatywnie wpłynie na badany współczynnik, uzyskując większą utratę przesyłanych pakietów,
niż w~przypadku o~uboższym tle radiowym (tutaj: teren miejski).

Główną treścią celem pracy jest rozdział~\ref{ch:experiment}, wprowadzający czytelnika w~aspekty praktyczne opisywanych
rozważań. Mając na uwadze wymienione wcześniej planowane doświadczenia, należy przygotować właściwy tor
pomiarowy, by zebrać dane, celem dalszej analizy. Do celów pracy, wykorzystano
gotowe komercyjnie dostępne zestawy uruchomieniowe \textit{P-NUCLEO-WB} oraz pomiarowy
\textit{X-NUCLEO-LPM01A}. Opierając się na wybranym sprzęcie, wprowadza w~szczegóły
tworzenia niezbędnych urządzeń wspomagających i~oprogramowania koniecznego
do przeprowadzenia badań.

Przedstawiana jest metodologia poszczególnych eksperymentów. Określone są warunki początkowe,
parametry techniczne badanych właściwości. Prezentuje się również metodologię oraz
nawiązuje do właściwych, wcześniej wymienionych, wzorów i~zależności, o które oparto
zebrane dane. Ostatecznie, prezentowane są wykresy nakreślające cechy zebranych pomiarów.
Wykresy, oprócz rzeczywistych danych, prezentują linie aproksymacyjne, celem określenia
tendencji wzrostów/spadków PER.

Finalnie, przedstawiane są wnioski zarówno z~przygotowań do przeprowadzenia eksperymentów
jak i~przede wszystkim wyciągane są należyte obserwacje i~ich konsekwencje, prezentując równocześnie 
dalsze kierunki rozwoju opisywanej tej pracy analizy.



\chapter{Komunikacja radiowa bliskiego zasięgu}
\lipsum[1-4]

\section{ZigBee}

ZigBee jest standardem transmisji bezprzewodowej zapewniający niskokosztową platformą
możliwą do zastosowania w elektronice użytkowej, automatyce domowej, wszelkiego rodzaju sensorach
(w~szczególności przemysłowych i~medycznych) jak również grach i~zabawkach.
Pierwsza specyfikacja opublikowana została w grudniu 2004 roku będąc ciągle aktualizowana,
z najnowszą jej wersją będącą datowaną na marzec 2017 roku~\cite{zigbee_alliance_zigbee_2017}.

Architektura ZigBee oparta została o IEEE 802.15.4. Definiuje ona fundamentalne zagadnienia:
\gls{PHY} i~\gls{MAC}. Warstwa fizyczna odpowiada za funkcjonowanie
radia, \gls{LQI}, transmisję danych i~odbiór pakietów poprzez łącze fizyczne. Definiuje dozwolone
częstotliwości działania, szerokości pasma, rodzaj modulacji i~dozwoloną przepustowość danych
wyrażonych w bitach na sekundę. Warstwa MAC odpowiada za komunikację w wyższych warstwach stosu.
Obejmuje to między innymi zarządzanie dostępem do kanałów, walidacja ramek danych, informację
zwrotną o~otrzymaniu i~przetworzeniu danych~\gls{ACK} oraz zapewnia odpowiednie
uchwyty celem umożliwienia wdrożenia mechanizmów zabezpieczeń.
Standard wprowadza również pojęcie topologii uwzględniając tym samym sposoby,
w~jakich można zorganizować sieć poszczególnych urządzeń. Definiowane są dwie
opcje połączeń: gwiazda, peer-to-peer. Topologia gwiazdy pozwala podłączenie wielu węzłów
uwzględniając fakt, iż komunikacja odbywa się za pośrednictwem koordynatora \gls{PAN},
będący tożsamy z \gls{FFD}. W~przypadku konfiguracji rówieśniczej, urządzenia mogą 
komunikować się dodatkowo między sobą, zapewniając możliwość ustanowienia innych struktur, m.in. 
Mesh. Standard wprowadza określenie \gls{RFD}, będące najczęściej urządzeniem o~prostej funkcjonalności 
niewymagającym dużych ilości danych do funkcjonowania, o~zredukowanej potrzebie na 
zasoby sprzętowe~\cite{ieee_p80215_working_group_ieee_nodate}.

\begin{figure}[!ht]
	\centering \includegraphics[width=0.99\linewidth]{zigbee_stack_architecture.png}
	\caption{Architektura stosu ZigBee. Źródło:~\cite{zigbee_alliance_zigbee_2017}}
	\label{rys:zigbee_stack_architecture}
\end{figure}

Specyfikacja ZigBee, opierając się na dokumentach IEEE 802.15.4, wykorzystuje częstotliwości
$868/915 MHz$ (w zależności od regionu Europa albo USA/Australia) oraz 2.4GHz~\cite{zigbee_alliance_zigbee_2017}.
Umożliwia tym samym transfer z przepustowością do 250~kbps~\cite{silicon_laboratories_ug10302_2021}.
Omawiany standard wprowadza swoje dodatkowe warstwy komunikacji do stosu: \gls{NWK} oraz~\gls{APL} --
Rysunek~\ref{rys:zigbee_stack_architecture}.
Warstwa aplikacji, będącą najwyższą w~hierarchi, składa się z wielu składowych. \gls{APS} odpowiada za
komunikację pomiędzy \gls{NWK} a~warstwami wyższymi. Oferuje między innymi parowanie urządzeń,
przekazywanie wiadomości, adresację, zajmuje się fragmentacją pakietów i~zapewnia niezawodny transport danych.
\gls{ZDO} w~głównej mierze odpowiada za wyszukiwanie urządzenia i~usług ZigBee~\cite{stmicroelectronics_an5506_2020, zigbee_alliance_zigbee_2017}.
Nadaje on również role urządzeniom sieci.

\begin{figure}[!ht]
	\centering \includegraphics[width=0.618\linewidth]{zigbee_topologies_an5506.png}
	\caption{Topologie sieci ZigBee. Źródło:~\cite{stmicroelectronics_an5506_2020}}
	\label{rys:zigbee_topologies_an5506}
\end{figure}

ZigBee wprowadza trzy główne definicje ról urządzeń rejestrowanych do wewnątrz sieci:
\begin{itemize}
\item \gls{ZC} -- węzeł odpowiadający za utworzenie i~utrzymywanie scentralizowanej sieci, dobór wymaganych parametrów, dodawanie nowych węzłów.
\item \gls{ZR} -- węzeł odpowiadający za przekazywanie danych, który również może przyjąć rolę urządzenia końcowego. 
\item \gls{ZED} -- węzeł końcowy który odbiera i wysyła dane bez możliwości ich routowania.
\end{itemize}

Warstwa sieci umożliwia adaptację trzech rodzajów topologii: gwiazda, drzewo i mesh -- Rysunek~\ref{rys:zigbee_topologies_an5506}.
Typ gwiazdy kontrolowany jest przez jednego koordynatora. Topologia drzewa pozwala zastosować hierarchiczne
sposoby routingu pakietów. Typ mesh z kolei pozwala na pełną komunikację peer-to-peer między węzłami~\cite{zigbee_alliance_zigbee_2017}.
Zestawienie poszczególnych warstw z~modelem referencyjnym OSI znajduje się na Rysunku~\ref{rys:zigbee_osi_comparison_an5506}.

ZigBee umożliwia wykorzystanie następujących metod routingu. Metoda oparta o tablicę trasowania\footnote{z ang. \textit{Routing Table}}
zakłada, iż każdy z~węzłów posiada strukturę przechowującą adresy kolejnych, otaczających go węzłów. Raz wysłana wiadomość,
będzie korzystać z~tej informacji, by przesłać pakiet do miejsca docelowego. W~przypadku niepowodzenia, pierwotny węzeł otrzyma
błąd, by ewentualnie podjąć dalszą decyzję o~ponownym wyznaczeniu trasy. Standard przewiduje wysyłanie również pakietów
przy wykorzystaniu rozgłoszenia z możliwością wyboru roli danego urządzenia. Możliwy jest również multicast. Ostatnią
opcją trasowania jest metoda wiele-do-jednego (źródła)~\cite{silicon_laboratories_ug10302_2021}.

\begin{figure}[!ht]
	\centering \includegraphics[width=0.618\linewidth]{zigbee_osi_comparison_an5506.png}
	\caption{Zestawienie warstw stosu ZigBee z modelem referencyjnym OSI. Źródło:~\cite{stmicroelectronics_an5506_2020}}
	\label{rys:zigbee_osi_comparison_an5506}
\end{figure}

ZigBee wprowadza termin \textit{profili}, będący kontraktem pomiędzy komunikatami wysyłanymi pomiędzy urządzeniami. Definiuje on
logiczną strukturę danych i zapewniając kompatybilność pomiędzy platformami różnych producentów. Cechą tą charakteryzują
się przede wszystkim profile publiczne zdefiniowane przez ZigBee Alliance. Poszczególni producenci mogą 
również opracować własnościowe, zamknięte struktury do tworzenia wewnętrznych sieci, gdzie kompatybilność pomiędzy
urządzeniami wielu producentów nie jest wymagana~\cite{zigbee_alliance_zigbee_2017, stmicroelectronics_an5506_2020, zigbee_alliance_zigbee_2017}.

\section{Thread}
Thread jest protokół to do zastosowań \gls{IoT} mający swe podstawy w standardzie IEEE 802.15.4.
Umożliwia on tworzenie rozwiązań o niskim zużyciu energii, przy jednoczesnej koncentracji na bezpieczeństwie opierając
adresację o powszechnie znany IPv6. Wybór IPv6 zapewnia płynną integrację z~powszechną infrastrukturą
i~Internetem włącznie. Rozwiązanie jest przez to elastyczne i~mniej podatne na starzenie się technologii.
Same natomiast produkty oparte o~Thread mogą być wdrożone na rynek szybciej dzięki powszechności
internetowych narzędzi deweloperskich. Pierwsza wersja specyfikacji została udostępniona
w 2014 roku i~pozostaje rozwijana do dziś~\cite{noauthor_thread_nodate}.

\begin{figure}[!ht]
	\centering \includegraphics[width=0.618\linewidth]{thread_stack_overview_ug10311.png}
	\caption{Stos Thread i odpowiadające mu specyfikacje RFC/IEEE. Źródło:~\cite{silicon_laboratories_ug10311_2022}}
	\label{rys:thread_stack_overview_ug10311}
\end{figure}

Thread wykorzystuje standard IEEE 802.15.4. Transmisja danych odbywa się na częstotliwości $2.4GHz$ z przepustowością
do 250~kbps. Protokół jest zoptymalizowany do wykorzystywania dużej liczby węzłów wchodzących w skład sieci~\cite{silicon_laboratories_ug10311_2022}.
Celem organizacji sieci w~spójny zbiór fizycznych i~logicznych obiektów, standard wprowadza następującą nomenklaturę.
Rolą węzła typu router jest przekazywanie pakietów w sieci oraz nadzorowanie dostępu do sieci, podczas gdy radio
takiego urządzenia ciągle jest w aktywne. \gls{ED} jest urządzeniem przeważnie komunikującym się z~jednym
routerem będącym jego rodzicem, nie przekazuje pakietów dla innych sieci i~możliwością deaktywacji radia
celem ograniczenia zużycia energii. Dodatkowo, definiuje się następujące typy urządzeń:
\begin{itemize}
\item \gls{FTD} -- urządzenie posiadające ciągle włączone radio, mapujące adresację IPv6
	\begin{itemize}
	\item Router
	\item \gls{REED} -- urządzenie, które można wykorzystywać jako router
	\item \gls{FED} -- urządzenie, którego nie można wykorzystywać jako router
	\end{itemize}
\item \gls{MTD} urządzenie przekazujące komunikaty do rodzica.
	\begin{itemize}
	\item \gls{MED} -- urządzenie, którego radioodbiornik zawsze pozostaje włączony. Nie wymaga periodycznego
	pobierania wiadomości z urządzenia-rodzica
	\item \gls{SED} -- urządzenie wzbudzające radioodbiornik okazjonalnie celem pobrania wiadomości.
	\end{itemize}
\end{itemize}
Standard przewiduje dodatkowe typy jak Thread Leader, będący dynamicznie i automatycznie wybieranym węzłem
zarządzający pozostałymi Router'ami w sieci. Border Router (router brzegowy) służy za bramkę
konwertujący komunikaty przesyłane wewnątrz sieci Thread do sieci zewnętrznych takich jak Internet~\cite{noauthor_node_2022}.

\begin{figure}[!ht]
	\centering \includegraphics[width=0.618\linewidth]{thread_topology_devices_threadgroup.png}
	\caption{Podstawowa topologia sieci Thread wraz z urządzeniami. Źródło:~\cite{thread_group_thread_2020}}
	\label{rys:thread_topology_devices_threadgroup}
\end{figure}

Dane w sieci przesyłane są w oparciu o~standard \textit{6LoWPAN}\footnote{IPv6 Over Low Power Wireless Personal Networks}.
Protokół ten został zoptymalizowany w ten sposób, by wysyłać maksymalną możliwą ilość danych z użyciem jednego
pakietu celem minimalizacji fragmentacji pakietów, redukując tym samym narzut na CPU i~zużycie energii.
Thread umożliwia stosowanie również protokołów znanych z sieci opartych o model TCP/IP. Tak więc
standard ten obsługuje m.in. \gls{ICMP}, \gls{UDP}, \gls{TCP}. Topologia tym samym również zależy od ilości
węzłów typu router, tworząc albo sieć gwiazdy, albo mesh. Thread obsługuje do 32-óch routerów, gdzie każdy
z~nich może obsłużyć do 511 \gls{ED}. Routing odbywa się na zasadach znanych z~\gls{IP}
wykorzystując w tym celu protokół zbliżony do \gls{RIP}, będący zoptymalizowany do wymagań
IoT pod względem zużycia energii~\cite{silicon_laboratories_ug10311_2022}.

\begin{figure}[!ht]
	\centering \includegraphics[width=0.618\linewidth]{thread_iso_comparison_ug10311.png}
	\caption{Thread w zestawieniu z modelem OSI. Źródło:~\cite{silicon_laboratories_ug10311_2022}}
	\label{rys:thread_iso_comparison_ug10311}
\end{figure}

Thread będący otwartym standardem może zostać wdrożony przez producenta samodzielnie bądź wybrać stos otwartoźródłowy --~\ref{rys:thread_iso_comparison_ug10311}.
Jednym z takich stosów jest OpenThread\footnote{\url{https://openthread.io/}} będący opracowany przez firmę Google.

Co warto dodać, Thread nie definiuje formatu danych jaki jest wysyłany pomiędzy węzłami. Innymi słowy,
warstwa aplikacji jest nieustandaryzowana~\cite{silicon_laboratories_ug10311_2022}. Obecnie trwają
prace nad opracowaniem wspólnego i~wolnościowego interfejsu. Jednym z takich projektów jest \textit{Matter}\footnote{\url{https://csa-iot.org/all-solutions/matter/}}.

\section{Bluetooth Low Energy}
\lipsum[1-8]

\section{Porównanie przedstawionych standardów} % jako podroździał
% historia jako delikatny wstęp
\url{https://www.silabs.com/documents/public/application-notes/an1142-mesh-network-performance-comparison.pdf}
\url{https://www.silabs.com/wireless/matter}
\lipsum[1-15]


% 1. Krótki rys historyczny standardu
% 2. Opis stosu + jakiś obrazek
% 2.1. Częstotliwości na których pracuej standard
% 3. Topologia, rodzaje węzłów

\section{Teoretyczne podstawy opisywanych badań}
\section{Teoretyczne podstawy zaprojektowanych eksperymentów}

Celem tego podrozdziału jest zapewnienie minimalnego, teoretycznego wstępu
wymaganego do przeprowadzenia właściwych badań. Poniższe punkty
prezentują definicje, wzory i symulacje. Zawarte zostaną również
hipotezy badawcze, które w~praktycznej dalszej części pracy
zostaną empirycznie zweryfikowane.

\subsubsection{Badanie zużycia energii}
Energooszczędność urządzeń elektronicznych ma szczególne znaczenie w~przypadku produktów
zasilanymi ogniwami. Celem optymalizacyjnym jest wydłużenie czasu użytkowania na pojedynczym
ładowaniu bądź do kolejnej wymiany baterii.

Przed przystąpieniem do fizycznego eksperymentu, wykonano przykładową symulację z~użyciem
oprogramowania firmy ST, będącej producentem wybranej platformy opisanej w~kolejnym rozdziale.
\textit{STM32CubeMX} będący zintegrowanym środowiskiem programistycznym zapewnia moduł
o~nazwie \textit{PCC}~\cite{noauthor_um1718_2022}. Umożliwia on symulowanie zużycia energii 
przez układ zgodnie ze zgrubnymi szacunkami inżyniera.

\begin{figure}[!ht]
	\centering \includegraphics[width=0.99\linewidth]{cube_pcc_advertising_1ms.png} 
	\caption{Symulacja zużycia energii przez urządzenie \gls{BLE} podczas ogłaszania}
	\label{rys:cube_pcc_advertising_1ms}
\end{figure}

Postępując zgodnie z dokumentacją, wykonano dwie symulacje. Rysunek~\ref{rys:cube_pcc_advertising_1ms}
odzwierciedla przewidywane zużycie energii urządzenia BLE dla domyślnie proponowanych nastaw sugerowane przez \textit{PCC}.
Krok trzeci jest etapem aktywacji radia rozgłaszającego komunikaty o~gotowości do połączenia. Szacowany
pobór prądu wynosi ok. 5mA, przy średniej 2.95mA. Jako główne źródło zasilania, do celów
symulacji, dobrano \enquote{\textit{wirtualną}} baterię 3.3V -- Li-SOCL2 (A3400). Umotywowane jest to wykorzystywaniem analogicznego źródła
jak podczas właściwego badania. Oprogramowanie sugeruje, iż bateria zostanie rozładowana po nieco ponad półtora
miesiąca użytkowania.

Analogiczną symulację przeprowadzono dla połączonego już urządzenia -- Rysunek~\ref{rys:cube_pcc_connected_master}.
Dla zadanych parametrów, oczekiwana długość życia układu będzie o~dwie godziny dłuższa niż wcześniej wspomniany
przypadek.

\begin{figure}[!ht]
	\centering \includegraphics[width=0.99\linewidth]{cube_pcc_connected_master.png} 
	\caption{Symulacja zużycia energii przez urządzenie \gls{BLE} podczas połączenia}
	\label{rys:cube_pcc_connected_master}
\end{figure}

Moduł \textit{PCC} niestety nie zapewnia wsparcia do symulacji węzłów sieci Mesh. Stawia się hipotezę,
iż zużycie energii będzie zależało od poszczególnie pełnionych ról przez dane urządzenie. Standard
przewiduje wykorzystywania zasilania bateryjnego przez węzeł \gls{LPN}. W przypadku takiej konfiguracji,
należałoby spodziewać się rezultatów podobnych do symulowanych. Przeprowadzając pomiary, wykorzystano
standardowe węzły, stale nasłuchujące za sygnałem będąc serwerem modelu \textit{Generic OnOff}. Spodziewane
jest znaczący wzrost zużycia energii właśnie ze względu na wymieniony fakt zaczerpnięty ze specyfikacji.

Zebrawszy dane chwilowych poboru natężenia prądu w czasie, należy wyliczyć właściwe parametry w~postaci
skonsumowanej energii i~mocy. Wykorzystuje się w celu oczywistą zależność fizyczną zaprezentowaną
wzorem~\ref{energy_equation}~\cite{skoro_marta_fizyka_1973}. Wzór uwzględnia okres trwania akwizycji danych
zdefiniowany w~docelowym podrozdziale~\ref{experiment:energy-consumption}.

\begin{equation} \label{energy_equation}
E_{\text{całkowita}} = U \cdot \int_{t=0[s]}^{t=50[s]} \mathrm{d}i \: \mathrm{d} t
\end{equation}

\begin{equation} \label{power_equation}
P = \frac{E_{\text{całkowita}}}{t}
\end{equation}

gdzie:

\begin{description}
\item $E_{\text{całkowita}} [J]$ - wykorzystana energia podczas 50s sekundowej sesji rejestracji danych
\item $P [W]$ - mocz użyta podczas 50s sekundowej sesji rejestracji danych
\item $U [V]$ - napięcie zasilania mikrokontrolera - $3.3V$ - stała
\item $\mathrm{d}i [A]$ - prąd w danej chwili
\item $\mathrm{d}t [s]$ - podstawa czasowa całkowania, 0.01s/interwał ($100Hz$) - stała 
\end{description}

\subsubsection{Badanie Packet Error Rate}\label{subsec:per_experiment}
Przed przystąpieniem do badań należy precyzyjnie zdefiniować wykorzystywaną terminologię.

Pakietem nazywamy pojedynczą porcję danych przetwarzaną na poziomie warstwy sieciowej modelu ISO OSI \cite{sa_tcpip_nodate}.
Warstwa ta umożliwia routing, adresowanie logiczne oraz przetwarzanie i~dostarczanie pakietów.

\begin{figure}[!ht]
	\centering \includegraphics[width=0.618\linewidth]{tcp_ip_szkola_programowania_naglowki_stopki.png} 
	\caption{Model ISO OSI wraz i odpowiadające mu nazwy porcji danych. Źródło: \cite{sa_tcpip_nodate}}
	\label{rys:iso_osi_model_nazwy_grup_danych}
\end{figure}

\gls{BLE} wprowadza własną nomenklaturę dla poszczególnych warstw sieciowych. Jest to o~tyle istotne, iż standard ten
nie zapewnia odpowiadających modelowi ISO OSI warstw jeden-do-jednego. Część z~tych warstw jest agregowanych w~zespół 
protokołów wyższych warstw - Rysunek~\ref{rys:agregacja_protokolow_ble}. %rysunek znajduje się w opisie stosu BT

Bazując na definicji modelu OSI oraz stosie BLE, pakietem można nazwać wiadomości będące możliwie blisko
warstwy \textit{\gls{LL}}. Podobną definicję prezentuje dokumentacja ST:
\enquote{Pakiet to pojedyncza oznaczona wiadomość wysłana przez jedno i~odebrana przez 
co najmniej jedno urządzenie.}\footnote{Tłumaczenie własne}~\cite{stmicroelectronics_pm0271_2021}

BLE Mesh dodatkowo wprowadza własne dodatkowe warstwy komunikacji, jak zostało to opisane w~podrozdziale~\ref{sec:ble}.
Każda z wymienionych warstw jest hermetyzowana za pośrednictwem dostarczanego przez producenta \gls{API}.
Middleware może tym samym zostać zamknięty przed dostępem, skutkując brakiem możliwości 
bezpośredniego nasłuchiwania pakietów w~warstwie \gls{LL}. Uwzględniając powyższe czynniki, \textit{pakietem} dla BLE Mesh nazywana będzie wiadomość najbliższa warstwie \gls{LL}.

Posiadając definicję pakietu, \gls{PER} możliwe staje się zdefiniowanie wzoru, a~zarazem znaczenia
głównego celu badań. PER jest miarą ilości błędnych pakietów w proporcji do wszystkich wysłanych 
pakietów, zgodnie ze zdefiniowanym wzorem~\ref{per_equation}.

\begin{equation}
\label{per_equation}
PER = \frac{s - r}{s} \cdot 100\%
\end{equation}

gdzie:

\begin{description}
\item[s] is ilość wysłanych pakietów
\item[r] is ilość odebranych pakietów
\item[s-r] - ilość niepoprawnych/błędnych pakietów
\end{description}

Eksperyment PER przeprowadzany w warunkach terenowych jest narażony na szereg czynników wpływających
na jakość transmisji. Projektowane doświadczenie starało się zawrzeć przynajmniej część z nich, co
zostało to opisane w podrozdziale~\ref{experiment:per}.

Uznając fakt przeprowadzania badań weryfikujących aspekty komunikacji Bluetooth opartej na częstotliwości
2.4GHz, naturalnym staje się dobór różnych środowisk, w których ten czynnik jest skrajnie różny.
Uzasadnia się to faktem interferencji fal. Warunki zurbanizowane zapewniają bogate tło radiowe,
w~szczególności oparte o częstotliwości będące wykorzystywane przez inne popularne technologie
takie jak WiFi. Zakłada się, iż w takim środowisku prawdopodobieństwo kolizji (interferencji)
jest większe aniżeli w środowisku względnie radiowo odizolowanym. Tym samym, jedną z~badanych
hipotez określa się jako dominujący wpływ tła radiowego w warunkach zurbanizowanych na wzrost
miary \gls{PER}. Analogicznie, w warunkach odizolowanych radiowych, oczekiwany jest mniejsze
tempo wzrostu PER wraz ze zwiększanym dystansem pomiędzy węzłami.

Czynnikiem który może mieć wpływ na rezultaty jest rozmieszczenie węzłów sieci. Zgodnie
z~opisem metodologii (podrozdział~\ref{experiment:per}), każde z urządzeń układane było
bezpośrednio na glebie. Konsekwencją takie stanu rzeczy jest wpływ zjawiska zwanego
rozchodzeniem się fali powierzchniowej. Zgodnie z definicją: \enquote{[antena] jest umieszczona na powierzchni
ziemi, jeśli zawieszono ją na wysokości mniejszej niż $\lambda$ nad powierzchnią}~\cite{szostka_fale_2006}. 
W~przypadku wykorzystania komunikacji Bluetooth, definicję spełnia każde położenie
urządzenia poniżej linii 12,5cm nad powierzchnią ziemi.

Nie mniej istotna na ostateczne efekty badań jest pogoda w postaci czynników temperatury
czy wilgotności. Zmiana w tych parametrach wpływa na wartości tłumienia sygnału.



\chapter{Część doświadczalna}
\section{Rozwiązanie techniczne}
\subsection{Zestaw uruchomieniowy P-NUCLEO-WB55}
\subsection{Zestaw pomiarowy X-NUCLEO-LPM01A}
\subsection{HAL i stos Bluetooth Low Energy}

\section{Przygotowanie eksperymentu}
\subsection{Oprogramowanie mikrokontrolera}
\subsection{Oprogramowanie PC}
\subsection{Pozostały sprzęt}

%%!!!!!!!!!!!!!!!!!!!!!!!!!!!!!!!!!!!!!!!!!!!!!!!!!!!!!!!!!!!!!!!!!!!!!!!!!!!!!!
%%%%%%%%%%%%%%%%%%%%%%%%%%%%%%%%%%%%%%%%%%%%%%%%%%%%%%%%%%%%%%%%%%%%%%%%%%%%%%%%
%% SECTION: Zużycie energii
%%%%%%%%%%%%%%%%%%%%%%%%%%%%%%%%%%%%%%%%%%%%%%%%%%%%%%%%%%%%%%%%%%%%%%%%%%%%%%%%
%%!!!!!!!!!!!!!!!!!!!!!!!!!!!!!!!!!!!!!!!!!!!!!!!!!!!!!!!!!!!!!!!!!!!!!!!!!!!!!!
\section{Zużycie energii}

%%%%%%%%%%%%%%%%%%%%%%%%%%%%%%%%%%%%%%%%%%%%%%%%%%%%%%%%%%%%%%%%%%%%%%%%%%%%%%%%
%% SUBSECTION: Metodologia badania
%%%%%%%%%%%%%%%%%%%%%%%%%%%%%%%%%%%%%%%%%%%%%%%%%%%%%%%%%%%%%%%%%%%%%%%%%%%%%%%%
\subsection{Metodologia badania}

%%%%%%%%%%%%%%%%%%%%%%%%%%%%%%%%%%%%%%%%%%%%%%%%%%%%%%%%%%%%%%%%%%%%%%%%%%%%%%%%
%% SUBSECTION: BT Low Energy - Usługa Heart Rate
%%%%%%%%%%%%%%%%%%%%%%%%%%%%%%%%%%%%%%%%%%%%%%%%%%%%%%%%%%%%%%%%%%%%%%%%%%%%%%%%
\subsection{BT Low Energy - Usługa Heart Rate}

\begin{figure}[!htb]
	\centering \includegraphics[width=0.99\linewidth]{power_ble_hr_amps.png}
	\caption{Charakterystyka czasowa poboru prądu dla BLE i usługi Heart Rate}
	\label{rys:power_ble_hr_amps}
\end{figure}

\begin{figure}[!htb]
	\centering \includegraphics[width=0.99\linewidth]{power_ble_hr_amps_usage_juxtaposition.png}
	\caption{Zestawienie zużycia prądu dla usługi Heart Rate w zależności od trybu działania}
	\label{rys:power_ble_hr_amps_usage_juxtaposition}
\end{figure}

%%%%%%%%%%%%%%%%%%%%%%%%%%%%%%%%%%%%%%%%%%%%%%%%%%%%%%%%%%%%%%%%%%%%%%%%%%%%%%%%
%% SUBSECTION: BLE Mesh - Model Generic OnOff
%%%%%%%%%%%%%%%%%%%%%%%%%%%%%%%%%%%%%%%%%%%%%%%%%%%%%%%%%%%%%%%%%%%%%%%%%%%%%%%%
\subsection{BLE Mesh - Model Generic OnOff}

Pomiary dla BLE Mesh uwzględniające dwa tryby działania: sieć w oczekująca na komunikaty oraz podczas działania aktywnego korzystania z Modelu Generic OnOff.

\begin{figure}[!htb]
	\centering \includegraphics[width=0.99\linewidth]{power_ble_mesh_amps.png} 
	\caption{Charakterystyka czasowa poboru prądu dla BLE Mesh i modelu Generic OnOff}
	\label{rys:power_ble_mesh_amps}
\end{figure}

\begin{figure}[!htb]
	\centering \includegraphics[width=0.99\linewidth]{power_ble_mesh_amps_usage_juxtaposition.png} 
	\caption{Zestawienie zużycia prądu dla BLE Mesh w zależności od trybu działania}
	\label{rys:power_ble_mesh_amps_usage_juxtaposition}
\end{figure}

\begin{figure}[!htb]
	\centering \includegraphics[width=0.99\linewidth]{power_ble_consumption_comparison.png} 
	\caption{Porównanie średniego zużycia energii pomiędzy BT Low Energy HRT i BLE Mesh}
	\label{rys:power_ble_consumption_comparison}
\end{figure}





%%!!!!!!!!!!!!!!!!!!!!!!!!!!!!!!!!!!!!!!!!!!!!!!!!!!!!!!!!!!!!!!!!!!!!!!!!!!!!!!
%%%%%%%%%%%%%%%%%%%%%%%%%%%%%%%%%%%%%%%%%%%%%%%%%%%%%%%%%%%%%%%%%%%%%%%%%%%%%%%%
%% SECTION: Packet Error Rate
%%%%%%%%%%%%%%%%%%%%%%%%%%%%%%%%%%%%%%%%%%%%%%%%%%%%%%%%%%%%%%%%%%%%%%%%%%%%%%%%
%%!!!!!!!!!!!!!!!!!!!!!!!!!!!!!!!!!!!!!!!!!!!!!!!!!!!!!!!!!!!!!!!!!!!!!!!!!!!!!!
\section{Packet Error Rate}

%%%%%%%%%%%%%%%%%%%%%%%%%%%%%%%%%%%%%%%%%%%%%%%%%%%%%%%%%%%%%%%%%%%%%%%%%%%%%%%%
%% SUBSECTION: Zależność PER względem odległości między węzłami
%%%%%%%%%%%%%%%%%%%%%%%%%%%%%%%%%%%%%%%%%%%%%%%%%%%%%%%%%%%%%%%%%%%%%%%%%%%%%%%%
\subsection{Metodologia badania}

\begin{figure}[!htb]
	\centering \includegraphics[width=0.618\linewidth]{per_two_nodes.pdf} 
	\caption{Dystans pomiędzy węzłami dla sieci dwóch mikrokontrolerów}
	\label{rys:two_nodes_setup}
\end{figure}

Rysunek ~\ref{rys:two_nodes_setup}

\begin{figure}[!htb]
	\centering \includegraphics[width=0.99\linewidth]{per_three_nodes.pdf} 
	\caption{Dystans pomiędzy węzłami dla sieci trzech mikrokontrolerów}
	\label{rys:three_nodes_setup}
\end{figure}

%%%%%%%%%%%%%%%%%%%%%%%%%%%%%%%%%%%%%%%%%%%%%%%%%%%%%%%%%%%%%%%%%%%%%%%%%%%%%%%%
%% SUBSECTION: Zależność PER względem częstości zapytań
%%%%%%%%%%%%%%%%%%%%%%%%%%%%%%%%%%%%%%%%%%%%%%%%%%%%%%%%%%%%%%%%%%%%%%%%%%%%%%%%
\subsection{Zależność PER względem częstości zapytań}

\begin{figure}[!htb]
	\centering \includegraphics[width=0.618\linewidth]{per_to_distance_under_100ms.png}
	\caption{Zależność PER od dystansu dla zapytań o częstości $\leqslant$ 100ms dla różnej liczby węzłów}
	\label{rys:per_to_distance_under_100ms}
\end{figure}

\begin{figure}[!htb]
	\centering \includegraphics[width=0.618\linewidth]{per_to_distance_under_100ms_different_envs.png} 
	\caption{Zależność PER od dystansu dla zapytań o częstości $\leqslant$ 100ms w wybranych środowiskach bez rozróżnienia na liczbę węzłów}
	\label{rys:per_to_distance_under_100ms_different_envs}
\end{figure}

\begin{figure}[!htb]
	\centering \includegraphics[width=0.99\linewidth]{per_to_distance_under_100ms_different_envs_and_nodes.png}
	\caption{Zależność PER od dystansu dla zapytań o częstości $\leqslant$ 100ms w wybranych środowiskach i liczbę badanych węzłów}
	\label{rys:per_to_distance_under_100ms_different_envs_and_nodes}
\end{figure}

\begin{figure}[!htb]
	\centering \includegraphics[width=0.99\linewidth]{per_to_distance_over_100ms_different_envs_different_ping_interval.png} 
	\caption{Zależność PER od dystansu dla zapytań o częstości >100ms w wybranych środowiskach}
	\label{rys:per_to_distance_over_100ms_different_envs_different_ping_interval}
\end{figure}


%%%%%%%%%%%%%%%%%%%%%%%%%%%%%%%%%%%%%%%%%%%%%%%%%%%%%%%%%%%%%%%%%%%%%%%%%%%%%%%%
%% SUBSECTION: Zależność PER względem odległości między węzłami
%%%%%%%%%%%%%%%%%%%%%%%%%%%%%%%%%%%%%%%%%%%%%%%%%%%%%%%%%%%%%%%%%%%%%%%%%%%%%%%%
\subsection{Zależność PER względem odległości między węzłami}

\begin{figure}[!htb]
	\centering \includegraphics[width=0.618\linewidth]{per_to_distance_over_100ms.png}
	\caption{Zależność PER od dystansu dla zapytań o częstości >100ms dla różnej liczby węzłów}
	\label{rys:per_to_distance_over_100ms}
\end{figure}

\begin{figure}[!htb]
	\centering \includegraphics[width=0.618\linewidth]{per_to_distance_over_100ms_different_envs.png} 
	\caption{Zależność PER od dystansu dla zapytań o częstości >100ms w wybranych środowiskach bez rozróżnienia na liczbę węzłów}
	\label{rys:per_to_distance_over_100ms_different_envs}
\end{figure}

\begin{figure}[!htb]
	\centering \includegraphics[width=0.99\linewidth]{per_to_distance_over_100ms_different_envs_and_nodes.png}
	\caption{Zależność PER od dystansu dla zapytań o częstości >100ms w wybranych środowiskach i liczbę badanych węzłów}
	\label{rys:per_to_distance_over_100ms_different_envs_and_nodes}
\end{figure}



\chapter{Podsumowanie}
\label{ch:podsumowanie}
Niniejsza praca zrealizowała postawione założenia zdefiniowane we wstępie -- rozdział~\ref{ch:intro}.
Wprowadziwszy definicję sygnału płynnie wprowadzono zagadnienia związane z~rozpatrywanymi
standardami komunikacji bezprzewodowej.

Zestawiając wprowadzone protokoły na szczególną uwagę zasługują dwa z nich: badany Bluetooth~5
(z~wyróżnieniem Mesh) oraz Thread. Opierając się na wiedzy pochodzącej z~powszechnie
dostępnej literatury, Thread osiągnął znacząco \textit{lepsze} parametry transmisji danych.
Korzyść ta rozumiana jest jako najwyższa przepustowość dla sfragmentowanych danych (Rysunek~\ref{rys:throughput_vs_hops_an1142})
oraz najmniejsze opóźnienia niezależnie od wielkości wiadomości (Rysunek~\ref{rys:latency_vs_payload_an1142}) spośród
testowanych rozwiązań.

Thread nie definiuje warstwy aplikacji, jak to umożliwiają bądź wymuszają pozostałe rozważane standardy.
Czyni to protokół uniwersalnym, niemalże ogólnego przeznaczenia niczym protokoły TCP/IP. Porównanie jest
również nie bez znaczenia, gdyż Thread opiera się właśnie na IPv6, co znów zmniejsza nachylenie
krzywej uczenia się tego standardu i umożliwia niemalże bezpośrednio wpięcie do Internetu i~rozwiązań opartych
o~publiczną chmurę obliczeniową. Protokół ten, będąc oparty o IEEE 802.15.4, nie jest energooszczędny
do tego stopnia, co protokół BLE (Rysunek~\ref{rys:energy_per_packet_dbm_10.4108}). 
Nie czyni go to jednak gorszym a raczej przypisuje go do innych
zastosowań. Potencjalne problemy może stwarzać niewielka popularność tego protokołu wśród oferowanych
na rynku produktów. Jest to niemniej powiązane ze względną nową architekturą danej specyfikacji.
Utrudnieniem, które bezpośrednio wpływa na kompatybilność produktów różnych producentów, jest brak
wprost zdefiniowanej warstwy aplikacji. To wyzwanie adresują projekty, takie jak 
Matter\footnote{\url{https://csa-iot.org/all-solutions/matter/}}
udostępniając specyfikację integrującą rozwiązania oparte o~Thread.

Bluetooth Mesh, pomimo rezultatów wywodzących się z~powszechnej literatury również spełnia swoje
oczekiwania, do których został zaprojektowany. Protokół, kładzie szczególny nacisk na organizację
i~hierarchię transmitowanych wiadomości w~warstwie. Można wystosować wręcz opinię, że został stworzony
do wysyłania \textit{tylko jednego} komunikatu, by wykonać konkretną akcję. Takie twierdzenie można
poprzeć zarówno sposobem routingu i wspomnianej hierarchii wiadomości (modele Mesh), ale również
z biorąc uwagę wyniki reprezentowane na Rysunku~\ref{rys:latency_vs_payload_an1142}. Pojedynczy wysłany
komunikat cechuje się nie gorszym opóźnieniem niż pozostałe badane protokoły. Uwzględniając energooszczędność
tego standardu, czyni go nadzwyczaj interesującą opcją dla konkretnych zastosowań, np. inteligentnego
oświetlenia.

Dalsza część pracy to ćwiczenie praktyczne polegające pomiarze zużycia energii. Zaprezentowano
charakterystyki poboru prądu w czasie dla różnych trybów funkcjonowania urządzenia dla dwóch różnych firmware'ów.
Pierwszy rozważany przypadek oparty o BLE \gls{HRT} jest zgodny z~przeprowadzoną symulacją -- 
Rysunki\ref{rys:cube_pcc_advertising_1ms} oraz~\ref{rys:cube_pcc_connected_master}. Wykonana zgrubna symulacja,
bazująca na domyślnych parametrach wskazywanych przez dokumentację,
wskazywała na średni pobór prądu elektrycznego na poziomie 2,95mA (moc: 9,8mW)~\cite{noauthor_um1718_2022}. Jest to co prawda niemal
5-krotna różnica względem uzyskanych pomiarów -- maksymalny pobór prądu 0,69mA (moc: 2,27mW) -- jest ona jednak na korzyść
eksperymentu. Sugeruje to błąd dokonanych nastawów symulatora.

Przypadek zużycia Mesh wskazuje na zużycie energii węzła klasycznego serwera \textit{Generic OnOff}. Węzeł tego
najczęściej włączony jest do stałego źródła zasilania. Stąd, porównanie z Rysunku~\ref{rys:power_ble_consumption_comparison},
sugeruje na konieczność porównywania węzła typu \textit{LPN}, który w zamyśle -- potwierdzonym z~oficjalną specyfikacją
-- funkcjonuje z~wykorzystaniem zasilania bateryjnego. Słusznym rozwinięciem tego badania jest wykorzystania
\textit{LPN} z~jednosekundowym odświeżaniem swojego stanu (zakładając model \textit{Sensors}). Energooszczędność
BLE i Bluetooth Mesh wprost wynika z optymalizacji wykorzystania radia, co udowadnia choćby porównanie
wyników zaprezentowanych na charakterystykach~\ref{rys:power_ble_hr_fastadv_only_amps} oraz~\ref{rys:power_ble_hr_low_power_adv_only_amps},
gdzie podczas rozgłaszania niskomocowego (zgodnie z~opisywanymi w~danym rozdziale założeniami), radio przez większość
czasu nie pracuje.

Eksperyment PER wskazuje na kilka interesujących zależności. Pierwszą z nich jest wydajność stosu \gls{BT} mikrokontrolera \textit{STM32WB55RG}.
Przed przystąpieniem do formalnego doświadczenia, przeprowadzono szereg prób z częstościami wykonywanych zapytań
znacząco mniejszymi od 100ms --~w~analogii do ciągu Fibonacciego testowano interwały 10, 20, 30, 50 i~80ms. Każdy z~nich wykazywał
tę samą cechę utraty pakietów, jak na przedstawionym Rysunku~\ref{rys:per_to_distance_under_100ms}. Przeważnie,
po kilku-kilkunastu sekundach pracy węzeł bliższy nie odpowiadał na jakiekolwiek komunikaty AT. Po zrestartowaniu tego urządzenia,
ponownie działało bez zarzutu, umożliwiając pobranie informacji o~odebranej liczbie pakietów z~węzła dystalnego.
Sugeruje to raczej problemy ze stosem i~ewentualnym wyciekiem pamięci uniemożliwiające poprawne funkcjonowanie sprzętu.
Jednym z~kroków do dalszych analiz jest weryfikacja kodu i~jego działania w czasie rzeczywistym. Narzędzia jakie debugger
czy valgrind\footnote{\url{https://valgrind.org/}} byłyby pierwszym krokiem ku weryfikacji zagadnienia.

Ostatnią, aczkolwiek najistotniejszą zbadaną zależnością, jest wpływ dystansu na \gls{PER}. Zgodnie ze zdefiniowaną
hipotezą w~punkcie~\ref{subsec:per_experiment} i~opisywaną później metodologią, oczekiwano dominującą rolę otaczającego
środowiska radiowego w~przeprowadzanym eksperymencie.










% wyszukanie granicy na której następuje przełamanie trendu i normalne funkcjonowanie uC

Poza wymienionymi już możliwościami dalszego rozwoju badań, sugeruje się kolejne, następujące. Uwzględniając fakt utraty pakietów
ze względu na krótki interwał odpytywań, oczywistym jest wyszukanie punktu przełamania, w którym omawiana zależność nie występuje.
Celem by było wyznaczenie minimalnego interwału, który nie powoduje utraty pakietów na niewielkich odległościach w~kontrolowanych
warunkach.



    % Bibliografia - musi być
    % Bibliography - must exist
    \bibliografia

    % Strony końcowe - można zakomentować, jeśli zbędne
    % Additional pages - comment out if not needed
    
    % Wykaz symboli i skrótów - patrz opis w tekście przykładowym
    \acronymslist
    % Spis rysunków
    \listoffigures
    % Spis tabel
    \listoftables
    % Załączniki (plik appendices.tex)
    \easyappendices
\end{document}
%%%%%%%%%%%%%%%%%%%%%%%%%%%%%%%%%%%%%%%%%%%%%%%%%%%%%%%%%%%%%%%%%%%%%%%%%%%

